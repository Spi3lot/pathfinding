\begin{flushleft}
	
	\subsection*{Abstract}

	   \subsubsection*{The Java School Administration Software - jSAS:}
	
        	The aim of the project is to achieve autonomous driving for a model car using a LiDAR sensor. For this, a model was trained in the simulation environment CARLA.

            Firstly, an overview of various sensor technologies such as ultrasound, radar, LiDAR, and cameras is provided. Special attention is given to LiDAR technology as it is used in our project. Furthermore, the discussion delves into the specific applications, advantages, and disadvantages of each technology. Subsequently, various approaches to sensor data fusion are explored, followed by the selection of the sensor for our project using the scoring method.

            The second chapter addresses the shortest path problem. Several different algorithms are described and re-implemented from scratch to evoke as intuitive an understanding as possible. Finally, all of them are juxtaposed, and their respective strengths and weaknesses are summarized.
            
            In the third chapter, various types of machine learning and
            reinforcement learning algorithms are presented. This includes a detailed examination of Model-Based Reinforcement Learning and its specifics. Additionally, a custom MBRL agent is implemented to showcase the advantages of model-based algorithms. Furthermore, discussion revolves around the optimization and further development of MBRL algorithms and agents.
            
            In the final part of the thesis, the approach of Model Free Reinforcement Learning is initially explained. Additionally, the Explore-Exploit dilemma and the Monte Carlo methods are described. Afterward, some MFRL agents are introduced, and their functioning is further highlighted. Finally, applications of MFRL are presented.


\end{flushleft}