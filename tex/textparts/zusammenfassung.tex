\begin{flushleft}
	
	\subsection*{Zusammenfassung}

    	\subsubsection*{Einsatz von LiDAR im autonomen Fahren:}

            Das Ziel der Diplomarbeit ist es, ein Modellauto mit Hilfe eines LiDAR-Sensors zum autonomen Fahren zu bringen. Dazu wurde ein Modell in der Simulationsumgebung CARLA trainiert.
            
            Zu Beginn wird ein Überblick über die unterschiedlichen Sensortechnologien, wie Ultraschall, Radar, LiDAR und Kamera gegeben. Besonders wird auf die LiDAR-Technologie eingegangen, da diese in unserem Projekt eingesetzt wird. Des Weiteren wird auf die individuellen Einsatzgebiete, Vorteile sowie Nachteile eingegangen. Danach folgen unterschiedliche Ansätze der Sensordatenfusion und die Auswahl des Sensors für unser Projekt mit Hilfe der Scoring Methode. 
             
            Das zweite Kapitel behandelt das Problem der Berechnung des kürzesten Pfades. Es werden einige verschiedene Algorithmen beschrieben und von Grund auf neu implementiert, um ein möglichst intuitives Verständnis zu erwecken. Zum Schluss werden alle Algorithmen miteinander verglichen und eine Übersicht über deren Vor- und Nachteile gegeben.
            
            Im dritten Kapitel wird ein Überblick über die verschieden Arten von Machine
            Learning Algorithmen und Reinforcement Learning gegeben. Dabei wird genauer auf Model Based Reinforcement Learning und dessen Details eingegangen. Es wird auch ein eigener MBRL Agent implementiert, um die Vorteile von Model-Based Algorithmen zu zeigen. Des Weiteren wird über die Optimierung und Weiterentwicklung von MBRL Algorithmen und Agents gesprochen.
            
            Im letzten Teil der Diplomarbeit wird anfangs der Ansatz des Model Free Reinforcement Learnings erklärt. Zusätzlich werden das Explore-Exploit-Dilemma und die Monte Carlo Methoden beschrieben. Danach werden auch einige MFRL-Agenten vorgestellt und deren Funktionsweise genauer erläutert. Am Ende werden noch Anwendungsgebiete von MFRL vorgestellt.

\end{flushleft}
